
\documentclass{article}

\begin{document}

\title{Tutorial for Phase Estimation}
\author{Rhys Hawkins}
\maketitle

\section{Introduction}

This tutorial provides a worked example for estimating phase and group dispersion
from ambient noise cross-correlations using the method described in

\begin{flushleft}
  Hawkins R. and Sambridge M., ``An adjoint technique for estimation of interstation phase and
  group dispersion from ambient noise cross-correlations``, BSSA, 2019
\end{flushleft}

Within this directory, there are bash scripts to run an example inversion of a ambient noise
cross-correlation to obtain path average dispersion/models. The main steps for this inversions
are

\begin{enumerate}
\item Create a reference model and dispersion curve
\item Estimate points along the phase velocity dispersion curve from peaks, troughs, and zero-crossings
  of the Noise Correlation Function.
\item Fit a 1D Path Average Earth model to the dispersion curves
\item Fit a 1D Path Average Earth model and amplitude model to the Noise Correlation function
\end{enumerate}

Each of these steps has a corresponding bash script for running the various stages of
the inversion process, i.e.

\begin{enumerate}
\item 00\_create\_reference.sh
\item 01\_create\_initial\_target\_phase.sh
\item 02\_fit\_initial\_target\_phase.sh
\item 03\_fit\_bessel.sh
\end{enumerate}

The first of these will generally only need to be run once, whereas the final three
are required for each station pair.

\section{Creating a reference model}

A reference model is required as a starting point for this inversion process. The file format used
to describe the reference model is a simple text file, a simple example

\begin{verbatim}
2
10.0 0 3.50
0.0 0 4.80
\end{verbatim}



\section{Input File Format}

The file format used by the programs here for observed ambient noise
cross-correlations is a simple text file format with some meta
information on the first two lines, followed by the noise correlation
function and empirical Greens function in the frequency domain.  An
example for the first 4 lines of one of the example data files is
shown below (these files are located in the example\_data
subdirectory).

\begin{verbatim}
  -21.096394000    65.109688000   -16.650196000    65.054375000   209.155962184
  2.000000 197    11.332828014     6.217725756 4097
  0.000000  2.144090067e-02  0.000000000e+00  1.942721063e-03  0.000000000e+00
  0.000244  2.136327242e-02 -2.339829045e-06  1.336357675e-02  7.392375819e-04
\end{verbatim}

The first line contains longitude and latitude of the two stations,
i.e. <lonA> <latA> <lonB> <latB>, followed by the interstation
distance in kilometres. The second line contains the sampling rate of
the signal in Hertz, the number of days of cross-correlations, the
signal to noise ratios for the acausal and causal signals, and finally
the number of points in the frequency domain representation of the
noise correlation functions.

The remaining lines contains the frequency domain representations of
the empirical Greens functions and noise correlation functions where
each line consists of <frequency>, <real EGF> <imaginary EGF> <real
NCF> <imaginary NCF>. The frequency domain representation is assumed
to be of a real signal and hence only the positive frequencies are
required.

For this inversion approach, the important parts of the data file
required are the interstation distance, the sampling rate and the
first and forth column (frequency and real component of the NCF). The
EGF columns are only used for comparing Group velocity predictions.


\end{document}

